\switchlanguage{en} % The abstract is supposed to be in English!

\thispagestyle{plain}

\section*{Abstract}
%Titel auf Englisch wiederholen.

%Es folgt die englische Version der Kurzfassung.

\switchlanguage{\lang} % Switch back to the document's default language.
Several decades of research have been dedicated to the representation of real interactions in virtual or remote environments. Haptic interfaces give the possibility to touch virtual objects and to produce sensations during texture exploration by sliding a hand-held tool across a textured surface. This process elicits perceptual information about the properties of a texture based on the data recorded from real interactions. This thesis describes mathematical models for synthesizing acceleration signals at different speeds of a user during the surface exploration. These acceleration signals are then used for generating audio data to present microscopic roughness of a texture. The application of Linear Predictive Coding (LPC) is shown for interpolating between signals. Furthermore, computing recorded signals' major frequencies to predict acceleration data is introduced as another possible method for switching between audio recordings when the user's scanning speed changes. For both cases, high correlations are obtained in human-subject experiments between the predicted and recorded data without creating perceptually noticeable artifacts.  The signals generated via LPC method were not discernible from the recorded signals with a percentage of 96 \% and the signals generated from major frequencies had the similarity rate of 76 \% to the reference signals considering the speed responsiveness, which shows promise of both method's usability for vibrotactile response. It is also shown with a speed test that user speed is correlated with surface friction and therefore the speed responsiveness can be displayed via 3-5 audio signals for the objects, which have a friction coefficient between 0.2 and 0.6. These results indicate that it is possible to render vibrations from recorded materials with signal analysis methods and can be displayed to users with a few audio signals to simulate real surface interactions considering speed responsiveness.






