\chapter{Conclusion}
\thispagestyle{empty}% no page number in chapter title page
The representation of real interactions in virtual or remote environments have been researched for several decades. Through haptic interfaces, it is possible to feel virtual objects and capturing high frequency vibrations when contacting the surface. Sliding a probe over a textured surface results in a rich collection of vibrations that elicit perceptual information about the model of the surface.

In order to display vibrotactile signals to represent the microscopic roughness of the surface, one can record the acceleration signal of a user scanning a texture at a specific speed and display it, when the user has the same velocity, whereas the force responsiveness is ignored based on past researches. However, this method is not easy to implement for every single speed level and not efficient for storing. Many different approaches have been suggested to overcome this problem and this thesis presented some methods for interpolating vibrotactile signals in order to mimic realistic vibrations with speed dependency. The presented methods are Linear Predictive Coding and major frequency analysis.

In the first half of the thesis, the teory was explained about how to apply the mathematical principles of LPC deciding the optimal number of filter coefficients and giving some insights into the method. As a result, it was expected that one can use this method for producing interpolated signals with an optimum number of 400 filter coefficients. Then, the other method, namely major frequency analysis, was introduced. In the course of finding the dominant frequencies, which have the rich and valuable information of a signal, there seemed to arise some challenges. Our first challenge was avoiding the beats with a selection of frequencies that have a threshold difference among each other. The observations with example signals brightened the fact that the threshold difference can not be the same in high and low frequencies. Therefore, several threshold values were selected and assigned to corresponding intervals. The second challenge was to find the number of frequencies to be selected in order to produce a perceptually similar signal to the reference. For that, an experiment was built as a part of Experiment I with 10 subjects and the optimum was pointed around 10 frequencies.

In chapter \ref{friction}, an experiment run was executed in a haptic showroom with different friction coefficients in order to find speed value distribution depending on friction. 

In chapters \ref{data} and \ref{tactile}, experiments were built in order to investigate whether the signals created via both methods were perceptually discernable from the recorded signals. Experiment I dealt with comparisons of original signals against and IFFT transformed versions, LPC synthesized



We then described an experiment to test whether



 represents the acceleration response
under specific probe-surface interaction conditions.

Finally, we compare the results of this process with real recorded acceleration
signals, and we show that the two correlate strongly in the frequency domain.

Scanning a textured surface with a tool generates rich high-frequency signals that demonstrate mainly microscopic roughness of a surface.

  This process elicits perceptual information about the properties of a texture based on the data recorded from real interactions. This thesis describes mathematical models for synthesizing acceleration signals at different velocities of a user during the surface exploration. These acceleration signals are then used for producing audio data to present microscopic roughness of a texture.  Furthermore, computing recorded signals' major frequencies to predict acceleration data is introduced as another possible method for switching between audio recordings when the user's scanning speed changes. For both cases, high correlations are obtained between the predicted and recorded data without creating perceptually noticeable artifacts.