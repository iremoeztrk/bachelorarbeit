\chapter{Conclusion}
\thispagestyle{empty}% no page number in chapter title page
The representation of real interactions in virtual or remote environments have been researched for several decades. Through haptic interfaces, it is possible to feel virtual objects and capturing high frequency vibrations when contacting the surface. Sliding a probe over a textured surface results in a rich collection of vibrations that elicit perceptual information about the model of the surface.

In order to display vibrotactile signals to represent the microscopic roughness of the surface, one can record the acceleration signal of a user scanning a texture at a specific speed and display it, when the user has the same speed, whereas the force responsiveness is ignored based on past researches. However, this method is not easy to implement for every single speed level and not efficient for storing. Many different approaches have been suggested to overcome this problem and this thesis presented some methods for interpolating vibrotactile signals in order to mimic realistic vibrations with speed dependency. The presented methods are Linear Predictive Coding and major frequency analysis.

In the first half of the thesis, the theory was explained about how to apply the mathematical principles of LPC deciding the optimal number of filter coefficients and giving some insights into the method. As a result, it was expected that one can use this method for generating interpolated signals with an optimum number of 400 filter coefficients. Then, the other method, namely major frequency analysis, was introduced. In the course of finding the dominant frequencies, which have the rich and valuable information of a signal, there seemed to arise some challenges. Our first challenge was avoiding the beats with a selection of frequencies that have a threshold difference among each other. The observations in a human-subject experiment brightened the fact that the threshold difference can not be the same in high and low frequencies. Therefore, several threshold values were selected and assigned to corresponding intervals. The second challenge was to find the number of frequencies to be selected in order to generate a perceptually similar signal to the reference. For that, an experiment was built as a part of Experiment I with 10 subjects and the optimum was pointed around 13 frequencies.

In chapter \ref{friction}, an experiment run was executed in a haptic showroom with different friction coefficients in order to find speed value distribution depending on friction. With these results, we were able to determine the number of intervals for displaying different audio and compare the signals at that probable speed of that material, which makes it possible to be compatible with real world. The results were as expected, the higher friction coefficient gets, the smaller is the mean speed value and the less intervals are needed to create speed dependent vibrotactile signals.

In chapters \ref{data} and \ref{tactile}, experiments were built in order to investigate whether the signals created via both methods were perceptually discernible from the recorded signals. Experiment I dealt with comparisons of original signals against IFFT transformed versions, LPC synthesized signals respectively. As a result can be said that the \emph{ifft} transformed signals were similar and in fact can be used as a reference signal by being a stable version of the original. The LPC Method was also justified because synthesized signals could mostly (96 \%) not be distinguished from the reference signal. The rest of Experiment I consists of the test, which is already described for finding the optimal number of frequency in major frequency method. 

Experiment II was carried out in a haptic showroom with 5 materials. Participants were asked to rate the realism of signals in respect to speed responsiveness according to the reference signals. Achieved results show promise of this method's usability but could be improved regarding frequency selection and threshold determination. All in all can be said that major frequency method enables users to feel realistic vibrations, which respond to their speed.

For future work, the method of major frequency can be improved and the experiments can be done by a extended range of materials and with more participants to get more stable results. As an alternative another human subject experiment can be carried out with LPC synthesized signals from which new signal is reproduced via its major frequencies, so that the quality of LPC method is enhanced.  







