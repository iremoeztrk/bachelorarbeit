
\chapter{Introduction}
\pagenumbering{arabic}%Ab hier, werden arabische Zahlen benutzt
\setcounter{page}{1}%Mit Abschnitt 1 beginnt die Seitennummerierung neu.
\thispagestyle{empty}
%Die Einleitung soll zum eigentlichen Themengebiet hinf�hren und die
%Motivation f�r die Arbeit liefern. Am Schlu� der Einleitung wird
%weiterhin noch eine �bersicht �ber die restliche Arbeit gegeben.

Scanning a textured surface with a tool generates rich high-frequency signals that describe mainly microscopic roughness of a surface. These captured vibrations are called vibrotactile or acceleration signals which is considered synonymous as in \cite{mouse18}.  To achieve full distal attribution, sensed vibrations must correspond to the user inputs in a physically appropriate manner \cite{Loomis92}.

A reasonable idea is that varying one's exploratory speed and normal force must significantly alter the
realism. However, studies (e.g. \cite{Culbertson15-WHC-Vibrations}) show that removing force responsiveness does not have a significant effect on the perceived  realism, whereas removing speed responsiveness is more salient to users. Therefore, vibration signals in this thesis include speed responsiveness and not force responsiveness.

Allowing texture vibrations to respond to user speed is a valuable part of creating realistic haptic textures, nonetheless can be a challenging and time consuming part of the implementation, if real recorded acceleration signals are displayed for every speed level. Some research \cite{romano10} elucidate mathematical models as an alternative solution for representing a vibration texture under specific probe-surface interaction conditions. 
 
This thesis has microscopic texture features as its center of focus and aims to evaluate the prediction results obtained from mathematical models, namely LPC and major frequency analysis, which are capable of reducing the size of stored models, to be able to display realistic vibrations at different speed levels through interpolated audio signals. 

