\chapter{Microscopic Roughness}
\thispagestyle{empty}% no page number in chapter title page
girizgah
\section{Methods for Interpolating Audio Signals}
Interpolate audio signals for different velocities.

There are methods: lpc and major frequency.

\subsection{Linear Predictive Coding}
The basic idea of Linear Predictive Coding (LPC) is to develop a transfer function that can predict each sample of a signal as a linear combination of the previous samples. It has applications in filter design and speech coding. 

We consider an IIR filter $H(z)$ of length n in the form $H(z)=[-h_1z^{-2}-h_2z^{-1}...-h_nz^{-n}]$.
Our acceleration data vector from PCA is called {$\vec{a}(k)$} in the following. The resulting prediction vector from our filter is $\vec{\hat{a}}(k)$. The residual signal {$\vec{e}(k)$} is the difference between these two signals.The transfer function $P$(z) is the result of the following equation:

\begin{equation}
\frac{\vec{e}(k)}{\vec{a}(k)}=1-H(z)=P(z)   
\end{equation}

It is possible to compute the residual at each step using the vector of filter coefficients $\vec{h}=[h_1 h_2 h_3 ... h_n]^T$:

\begin{equation}
\vec{e}(k)=a(k)-\hat{a}(k)=a(k)-\vec{h}^T\vec{a}(k-1)   
\end{equation}





\section{Beispiel f�r eine Abbildung}
\begin{figure}[h!]
  \begin{center}
        \includegraphics[width=4cm]{TUMLogo_oZ_Outline_schwarz_CMYK}
    \caption{Beispiel f�r eine Beschriftung.}
    \label{fig:ToUseWithReference}
  \end{center}
\end{figure}

\begin{equation}
mRG = {\beta \cdot \sum_{k=1}^K \sum_{l=1}^L \hat{\textbf{X}}(k,l)}
\end{equation}

Durch die \texttt{\bslash label} kann auf die Bilder mit
\texttt{\bslash ref} verwiesen werden
(z.B.~Abbildung~\ref{fig:ToUseWithReference}).

\section{Beispiele f�r Referenzen}
Die Literaturhinweise werden im Text z.B.\ folgenderma�en verwendet:\\
``..., wie in \cite{eberspaecher97} gezeigt, ...'' oder ``... es gibt
mehrere Ans�tze \cite{arnaud99,griswold90} ...''

\section{Schrifttypen}
Als Schrifttyp wird Arial oder Roman empfohlen. Bitte beachten, da�
Gr��en und Einheiten eine eigene Schreibweise haben:
\begin{description}
\item[Kursivschrift:] physikalische Gr��en (z.B.~$U$ f�r Spannung),
  Variablen~(z.B.~$x$), sowie Funktions- und Operatorzeichen, deren
  Bedeutung frei gew�hlt werden kann (z.B.~$f(x)$)
\item[Steilschrift:] Einheiten und ihre Vors�tze (z.B.~kg, pF),
  Zahlen, Funktions- und Operatorzeichen mit feststehender Bedeutung
  (z.B.~sin, lg)
\end{description} 

\clearpage

\section{Archivierung}
F�r die Archivierung sind alle Dateien der Arbeit (auch der Vortr�ge)
dem Betreuer zur Verf�gung zu stellen.  Weiterhin soll noch ein
\BibTeX-Eintrag der Arbeit erstellt werden (die Felder in eckigen
Klammern sind dabei auszuf�llen):
\begin{verbatim}
@MastersThesis{<Nachname des Autors><Jahr>,
  type =         {<Art der Arbeit>},
  title =        ,
  school =       {Institute of Communication Networks~(LKN),
                  Munich University of Technology~(TUM)},
  author =       {<Nachname des Autors>, <Vorname des Autors>},
  annote =       {<Nachname des Betreuers>, <Vorname des Betreuers>},
  month =        {<Monat>},
  year =         {<Jahr>},
  key =          {<Mehrere Suchschl�ssel>}
}
\end{verbatim}
