\chapter{Haptic texture rendering}
\label{haptic}
\thispagestyle{empty}% no page number in chapter title page

The human haptic perception system relies on kinesthetic and cutaneous sensory information provided by several receptors during probe-surface interactions \cite{leder09}.
 
The kinesthetic sense focuses on the perception of forces and torques acting on the human
body. Kinesthetic stimulation is sensed by mechanoreceptors located in the muscles. Kinesthetic mechanoreceptors include muscle spindles in parallel with muscle fibers and Golgi tendon organs in series with them at the connection with the skeleton \cite{rahul15}. In some mainly kinesthetic tasks, the tactile sense only indicates a contact.

Tactile perception is stimulated by cutaneous receptors. There are four kinds of mechanoreceptors in the glabrous skin: The Merkel cells (Braille dots and sharp edges), Meissner corpuscles (low-frequency vibrations), Pacinian corpuscles (high frequency stimuli), Ruffini endings (still unknown).

Friction, hardness, macroscopic roughness, microscopic roughness and thermal conductivity are the main dimensions to build haptic texture models. All of these aspects should be well considered in order to increase immersion into a virtual environment. In the following of this thesis, microscopic roughness is our main focus as a vibrotactile feature.

\section{Roughness}

As an abstract feature roughness plays an important role in haptic signal perception.The roughness dimension may be divided into two dimensions: macro and fine (micro) roughness as in \cite{okamoto13}.

\subsection{Macroscopic Roughness}

Due to the duplex nature of roughness,  macroscopic roughness should strictly be considered in order to understand fine roughness. Coarse roughness is mainly represented by the ``uneven'', ``relief'', or ``voluminous'' labels and mediated by different mechanisms from micro roughness. For coarse surface roughness, the spatial  distribution of SAI units is related to roughness perception \cite{okamoto13}.

\subsection{Fine Roughness}

In the mechanism of fine roughness perception, FAI and FAII units contribute and is mainly represented by the ``rough" label. During surface-tool or surface-finger sliding motions, high frequency vibrations can be extracted using an accelerometer mounted on a human finger. Microscopic roughness impressions can be characterized by these vibrations. The macro and fine roughness dimensions can be separated according to the mentioned aspects but can intersect each other during the process of perception.  Figure \ref{fig:texture} illustrates the process of texture production with an accelerometer.
\begin{figure}[h!]
	\begin{center}
		\includegraphics[width=14cm]{texture}
		\caption{Mechanism of texture production. A hand-held tool is used to stroke a textured surface. An acceleration sensor signal mounted on the tool measures the response of the hand-tool system as it hits surface features. Figure reproduced from \cite{rahul15} }
		\label{fig:texture}
	\end{center}
\end{figure}

The recorded acceleration signals depend on the scan speed and force, nonetheless we ignore the force dependencies as mentioned in the introduction part. Thereby, our approach in this thesis is generating vibrotactile signals based on the recorded acceleration data for different absolute scan speed. They are then displayed via a voice coil actuator. 


\chapter{Interpolation of vibrotactile signals}
\label{interpolation}

Priori model designs have recently been evolved into data-driven haptic textures. The apparent concept is simply displaying recorded texture acceleration signals, where the exploratory speed dependency is disregarded. However, this is not a sufficient way of representing acceleration response. Therefore, user speed should be incorporated into the feedback signal.

As in \cite{refined12} is described, there are some methods of relating speed variable to tactile signals such as switching between recordings according to speed, which may however be perceptually detectable. Another way is building a function depending on speed that gives a weighted average of recorded acceleration signals as output. The problem that may occur here is that the frequency content is not preserved due to the probable constructive and destructive interference between signals. These aspects create the idea of data-driven texture models.

In the following, we analyze two methods for synthesizing vibrotactile signals without creating noticeable artifacts. The first method is synthesizing two recorded acceleration signals under different speed conditions through LPC method and interpolating between them by linear interpolation of filter variables. The second one is reproducing these two signals from their major frequencies and interpolating between them to generate new signals.

In the following, we go deep into both of the principles and how to apply them to produce interpolated signals.

\section{Linear predictive coding}
\label{lpcmethode}
The basic idea of Linear Predictive Coding (LPC) is to develop a transfer function that can predict each sample of a signal as a linear combination of the previous samples. It has applications in filter design and speech coding. 

We consider an IIR filter $H(z)$ of length n in the form $H(z)=[-h_1z^{-2}-h_2z^{-1}...-h_nz^{-n}]$.
Our acceleration data vector from PCA is called {$\vec{a}(k)$} in the following. The resulting prediction vector from our filter is $\vec{\hat{a}}(k)$. The residual signal {$\vec{e}(k)$} is the difference between these two signals.The transfer function $P(z)$ is the result of the following equation:

\begin{equation} \label{eq:transferfunction}
\frac{\vec{e}(k)}{\vec{a}(k)}=1-H(z)=P(z)   
\end{equation}

It is possible to compute the residual at each step using the vector of filter coefficients $\vec{h}=[h_1 h_2 h_3 ... h_n]^T$:

\begin{equation} \label{eq:residual}
\vec{e}(k)=a(k)-\hat{a}(k)=a(k)-\vec{h}^T\vec{a}(k-1)   
\end{equation}

At this step, we aim to find the minimum value of the residual function $e(k)$. We are able to reduce the problem to Wiener-Hopf equation by a cost function based on mean-square error. The Wiener-Hopf equation can be solved by Levinson-Durbin \cite{durbin} algorithm, so that we can obtain our optimal filter vector $\vec{h_0}$. 

To synthesize new signals, we use a white noise signal {$\vec{e_g}(l)$} as input, which is filtered with $1/P(z)$, in order to generate our desired response {$\vec{a_g}(l)$}. For a better overview, we can rewrite the equations (\ref{eq:transferfunction}) and (\ref{eq:residual}) as follows:

\begin{equation}
\frac{\vec{a_g}(l)}{\vec{e_g}(l)}=\frac{1}{1-H(z)} =\frac{1}{P(z)} 
\end{equation}

\begin{equation}
a_g(l)=e_g(l)+\vec{h}^T\vec{a_g}(l-1) 
\end{equation}

Figure \ref{fig:lpc-blockdiagram} illustrates each of analyzing and synthesizing processes via block diagram. The value {$\vec{e_g}(l)$} is a randomly generated Gaussian white noise but its average signal power must be equal to that of the average signal power remaining in the residual, $P\{\vec{e}(k)\}$ after filter optimization.

\begin{figure} 
	\begin{center}
	\subfigure[Block diagram for prediction of the next contact acceleration $a(k)$
	given the recorded series $a(k)$ with residual $e(k)$.]{\includegraphics[width=0.49\textwidth]{lpcsynthesis}} 
	\subfigure[Block diagram for the synthesis of an acceleration signal $a_g(l)$
	from the appropriately scaled white noise input $e_g(l)$.]{\includegraphics[width=0.49\textwidth]{lpc}} 
	\caption{Signal generation through LPC principle. Figures reproduced from \cite{romano10}.} 
	\label{fig:lpc-blockdiagram}
	\end{center}
\end{figure} 

The definition of power is as in the following equation:

\begin{equation}
P\{\vec{a}(l)\}=\frac{1}{N}\sum_{n=0}^{N-1}|a(n)|^2
\end{equation}

\begin{figure} [h!]
	\begin{center}
		\includegraphics[width=0.70\textwidth]{error2} 
		
		\caption{The RMS Error of synthesized signals via LPC calculated with different coefficient orders. The original signals are recorded acceleration data at 5cm/s of materials coak, steel wool and fine foam. The chosen threshold is 400 as a result of this plot, where additional coefficients have minimal benefit.} 
		\label{fig:error2}
	\end{center}
\end{figure} 

This is equivalent to signal variance $\sigma�$, because our signals are zero-mean signals. Now, we have to determine the order of our prediction filter, which affect the accuracy of the prediction. The higher we choose the order, the smaller the residual gets. It means we have a better prediction with higher orders, but then the calculation gets  more complicated. It is possible to calculate the success of the synthetic result with a cost function defined as the RMS error as follows:

\begin{equation} \label{costfunction}
C\{\vec{a_g}(l)\}=\frac{RMS(DFT_s\{\vec{a}(l)\}-DFT_s\{\vec{\hat{{a_g}}}(l)\})}{RMS(DFT_s\{\vec{a}(l)\}}
\end{equation}


Using this equation, where $DFT_s\{\vec{a}\}$ represents the discrete Fourier transform of vector $\vec{a}$, it is possible to obtain the optimal order of the filter. As shown in Figure \ref{fig:error2}, the higher a coefficient order is chosen, the smaller gets the RMS error calculated via given formula, whereas a major change in error value can not be detected among the orders higher than 400. That is why in this thesis 400 is chosen as the optimal order for the best quality of results.


Now that we have generated our prediction filter with two unique variables $\vec{h}$ vector and $e_g(l)$, it comes to interpolate between our synthesized signals to create new signals. Bilinear interpolation of both the vector $\vec{h}$ and $e_g(l)$ of two signals in different velocities and applying these new values to our prediction filter result in new synthesized signals, so that signal data for audio signals at different force and velocities are created. For the interpolation of filter coefficients, we first convert them to line spectral frequencies (LSFs) to ensure stability during rendering \cite{modeling14}. Figure \ref{fig:lpc-original} shows an example signal and its LPC synthesized version both in time and spectral domain. 

\begin{figure} [htbp]
	\begin{center}
		\subfigure{\includegraphics[width=0.49\textwidth]{gr_doppelt}} 
		\subfigure{\includegraphics[width=0.49\textwidth]{gr_lpc}} 
		\caption{The plot left shows the acceleration data recorded during interaction with a granite tile and its spectrum. The LPC synthesized version of the signal with the coefficient order 400 is shown on the right side. Both signals maintain the same spectral characteristics, whereas they differ from one another in the time domain.}
		\label{fig:lpc-original}
	\end{center}
\end{figure} 

The general feel of haptic textures is governed by their spectral signature \cite{haptography}. As in the figure \ref{fig:lpc-original}, the recorded and synthesized data differ on time-domain, but their spectrum are quite similar. So it is expected that they will feel the same to a user. Experiment I in chapter \ref{data} is going to deal with whether these correlations are strong enough to fool the human sense of touch.

\section{Signal Generation from Major Frequencies}
\label{majorfrequency}
The other method for signal generation is using rich and valuable information of signals' dominant frequencies, which has the highest amplitudes. The frequency of the vibration must change as the users change their force and so that their velocity. This is one of the realistic methods for interpolation between signals recorded under different velocities.

At first we should determine the number of the frequencies we are going to deal with for synthesizing new signals. This is done as a part of Experiment I in chapter \ref{data} according to the participants' feedback. For our case we set 30 as the maximum number of selected frequencies and 2 as the minimum. The optimum will be evaluated with the aid of the first experiment.

\begin{figure} [htbp]
	\begin{center}
		\subfigure[30-31 Hz]{\includegraphics[width=0.30\textwidth]{schwebung1}} 
		\subfigure[30-32 Hz]{\includegraphics[width=0.30\textwidth]{schwebung2}} 
		\subfigure[30-33 Hz]{\includegraphics[width=0.30\textwidth]{schwebung3}} 
		\subfigure[30-34 Hz]{\includegraphics[width=0.30\textwidth]{schwebung4}} 
		\caption{The plotted signals are generated as a summation of two periodic signals, which have the frequencies given below respectively. The beats wave is clearly noticeable in the first two signals, whereas it starts to differ from the beats in (c) and (d).}  
		\label{fig:schwebung}
	\end{center}
\end{figure} 


In order to find the frequencies with highest amplitudes, we calculate the discrete Fourier transform of the two recorded data and then select highest amplitudes of the transformed signals. It is important here to ensure that selected frequencies should have a certain distance to each other, because the superposition of two pure tones with slightly different frequencies can lead to beats. The resulted wave is observed as a very low frequency of $\delta f$ and can be felt by the human tactile or auditory sense as if there is another stimulation \cite{beats}. To avoid this phenomenon, we remove frequencies among selected ones with a small difference.

The threshold difference that the frequencies should have among each other is not the same in low and high frequency range as described in \cite{rahul15}. In this thesis the intervals and their threshold differences are determined according to an artificial signal generated by two frequencies, which are selected closer step by step (see Figure \ref{fig:schwebung}) and the experiment results in \cite{beats}, which allow us to find appropriate frequency intervals and their threshold frequency to perceive the beats sensation. As a result, the major frequencies are selected according to the tabular \ref{tab:threshold}.  

\begin{table}[htbp]
	\caption{The frequencies are selected from given intervals and have the threshold difference among each other.}
	\begin{center}	
		
		\begin{tabular}{||c c||} 
			\hline
			Frequency interval in Hz & Difference threshold in Hz \\ [1ex] 	
			\hline\hline
			0-50  & 3   \\ 
			\hline
			50-100 & 4  \\
			\hline
			100-150 & 7  \\
			\hline
			150-250 & 10  \\
			\hline
			250-500 & 15 \\  
			\hline
			500-1400 & 120 \\  
			\hline
		\end{tabular}
		\label{tab:threshold}
	\end{center}
\end{table}


In Figures \ref{fig:co1_1} and \ref{fig:co1_2}, an example of coak material is shown to illustrate how the artificial signal is generated from the recorded data. The signal is recorded at a very slow scanning speed (50mm/s) on coak and the obtained three-axis time-domain signals are reduced to one-axis using DFT321 \cite{romano12}. The generated signal from selected frequencies is in Figure \ref{fig:co1_3} both in time and spectral domain. 

\begin{figure} [htbp]
	\begin{center}
	\includegraphics[width=0.70\textwidth]{co1_1} 
	 
	\caption{The depicted signal is recorded during an interaction with coak at a speed of 40 mm/s. The first plot shows the reduced one-axis signal in time domain and second plot shows spectral properties of the signal.} 
	\label{fig:co1_1}
		\end{center}
\end{figure} 

\begin{figure} [htbp]
	\begin{center}
		
	\includegraphics[width=0.70\textwidth]{co1_2} 

\caption{As shown in this example, the red points show the 30 selected frequencies and the black ones demonstrate the frequencies, which have amplitudes higher than the half of the maximum amplitude of the signal. Thus, the upper side of the spectral domain is shown with selected frequencies for a better overwiev.} 
\label{fig:co1_2}
\end{center}
\end{figure} 

\begin{figure} [htbp]
	\begin{center}
		\subfigure{\includegraphics[width=0.70\textwidth]{co1_time}} 
		\subfigure{\includegraphics[width=0.70\textwidth]{co1_spectrum}} 
	\caption{First plot is the time domain of the generated signal via the frequencies in the second plot, which were chosen in Figure \ref{fig:co1_2}.}
	\label{fig:co1_3}
	\end{center}
\end{figure} 


The selected frequencies of the recorded signals are utilized as skeleton of new signals to be generated. We determine the angle and the amplitude of each frequency before the generation process. We synthesize new signals by applying linear interpolation to the selected frequencies' amplitudes (see Figure \ref{fig:major}) and new signals are generated according to the following equation:


\begin{equation} \label{synthesize}
X_S(maxF(k))=maxA(k)*exp(maxP(k)*i)
\end{equation}

\begin{figure} [htbp]
	\begin{center}
		\subfigure[First generated signal from recorded data during interaction with coak at speed 40mm/s reproduced from given frequencies below.]{\includegraphics[width=0.49\textwidth]{coak1}} 
		\subfigure[An interpolation result between signals in (a) and (d).]{\includegraphics[width=0.49\textwidth]{coak5}} 
		\subfigure[Another result of the interpolation. Combination of the frequencies can be seen in the spectral domain plot below.]{\includegraphics[width=0.49\textwidth]{coak7}} 
		\subfigure[The reproduced signal from recorded data at speed 400mm/s reproduced from given frequencies below.]{\includegraphics[width=0.49\textwidth]{coak10}} 
		\caption{Signal interpolation through frequencies with highest amplitudes.} 
		\label{fig:major}
	\end{center}
\end{figure} 


where $maxA(k)$ represents the amplitude of selected frequencies and $maxF(k)$ the frequency, $maxP(k)$ represents the phase of selected frequencies with k as the selected frequency order. Signal $X_S$ is the generated spectrum, which needs to be transformed into the time domain via IFFT to get the estimated signal as in Figure \ref{fig:co1_3}.  

Finally in both methods, we generate audio signals by $audiowrite$ function in Matlab, which are displayed during interactions with virtual surfaces.

\chapter{Friction Dependence of Speed}
\label{friction}

Surface stickiness compels the users to apply a lateral force during surface interactions. The friction coefficient is usually calculated as the ratio
of the required dragging force of a sensor to the pressure, or
normal force \cite{mouse18}. 

\section{Speed Test}
In the following, we use a Phantom Omni device to feel a surface with different friction conditions, to be able to create a speed scala for some friction coefficients. Our aim is demonstrating how user speed varies depending on friction coefficients. We then use these results to evaluate speed intervals to be created in the following of this thesis, in which different audio data is displayed.

The test is carried out with an object in a haptic virtual environment, which has no friction at the beginning. The scanning speed is detected and saved in every 0.2 seconds, while the user explores the surface. Approximately after one minute, system automatically changes the friction and brings it to one level higher. The friction coefficients and the results are shown in Figures \ref{fig:boxplot} and \ref{fig:friction}.

\begin{figure} [htbp]
	\begin{center}
		\includegraphics[width=0.70\textwidth]{boxplot} 
		
		\caption{Boxplot between friction coefficients of a virtual surface and speed values detected in every 0.2 seconds.} 
		\label{fig:boxplot}
	\end{center}
\end{figure} 

\begin{figure} [htbp]
	\begin{center}
		\subfigure[]{\includegraphics[width=0.40\textwidth]{hist1}} 
		\subfigure[]{\includegraphics[width=0.40\textwidth]{hist2}} 
		\subfigure[]{\includegraphics[width=0.40\textwidth]{hist3}} 
		\subfigure[]{\includegraphics[width=0.40\textwidth]{hist4}} 
		\subfigure[]{\includegraphics[width=0.40\textwidth]{hist5}} 
		\subfigure[]{\includegraphics[width=0.40\textwidth]{hist6}} 
		\subfigure[]{\includegraphics[width=0.40\textwidth]{hist7}} 
		\caption{Histograms of each 7 friction coefficients with 240 speed samples saved in every 0.2 seconds during virtual surface interaction through a haptic device.The bin width is selected as 4 in each plot.}  
		\label{fig:friction}
	\end{center}
\end{figure} 


\section{Evaluation}

Figure \ref{fig:boxplot} presents the mean speed value and the common speed interval for each coefficient order. As expected, both the speed interval and the mean value get smaller with increasing friction coefficient. Figure \ref{fig:friction} gives us hints to be able to analyze how many intervals there are to display different audio data to simulate real surfaces.

In plot (a), nine bins are shown, where the last two can be ignored due to the small number. So in that case, we would choose seven intervals for changing audio as the user speed varies, on the other hand in real life, textures always have a friction above zero, so this example depicts an unnatural scenario. When we look at the most common coefficients like 0.2 and 0.4, which are also relevant for the experiments in following chapters, we recognize that, it is possible to work with less than seven intervals for vibrotactile response. In the case of 0.2, five bins are remarkable as peaks and the rest can be part of the last interval. For the friction 0.4, we can assign four intervals and for 0.6 three audio data are quite sufficient. The fine roughness of objects with 0.8 and 1.0 friction coefficient can be displayed with two signals and for the last and extreme case one audio is enough already according to the presented results, so that there is no need to interpolate the recorded data.



\chapter{Data-Driven Methods for Tactile Signal Evaluation}
\label{data}

The perception of displayed haptic
information typically varies across different human subjects \cite{leder09}. Therefore,experiments with human participants and their feedbacks constitute a fundamental component of the development in haptics.

As mentioned in chapter \ref{interpolation}, we analyze two methodologies of synthesizing vibrotactile signals for rendering fine roughness feedback: LPC and utilizing major frequencies of records. This chapter describes a human-subject experiment to evaluate how perceptually close our synthesized vibrotactile signals to the recorded acceleration signals in a haptic environment.  

\section{Experiment I}

\subsection{Subjects and Setup}

To accomplish a stable evaluation, ten volunteer human subjects, 3 female and 7 male, participated in the experiment at separate times. Their
ages ranged from 19 to 29, with an average of 23 years. The subjects were all right-handed with limited experience with haptic devices. None
of them reported having any ailments that would affect the experiment.

We used the voice coil actuator model NCC01-04-001-1X by H2W Technologies in the experiment to display vibrotactile signals. 5 objects were relevant for this experiment as shown in the tabular \ref{tab:materials}.
\begin{table}[htbp]
		\caption{Materials in the experiment given with their approximated friction coefficients.}
\begin{center}	
	
	\begin{tabular}{||c c||} 
		\hline
		Material & Friction \\ [1ex] 	
		\hline\hline
		Coarse foam & 0.6   \\ 
		\hline
		Fine foam & 0.6  \\
		\hline
		Coak & 0.2  \\
		\hline
		Granite Tile & 0.2  \\
		\hline
		Steel wool & 0.4 \\  
		\hline
	\end{tabular}
	\label{tab:materials}
\end{center}
\end{table}



\subsection{Procedure}

The subject sat at a table in front of the voice coil actuator. In each experiment run, the subject compared a synthesized signal with the reference while perceiving the response with the right index finger. The displayed reference signal was for each material a recorded acceleration data at the most probable speed according to the results in Figure \ref{fig:friction} in chapter \ref{friction} and the compared synthesized signal was reproduced from this data both via LPC and major frequency method. 

In addition to LPC and major frequency methodologies, the signal obtained by converting the initial signal from time domain to frequency domain via DFT and then converting back from the frequency domain to the time domain via IFFT was also displayed to be compared with the reference by the participants. The intent here is to remove possible peculiarities that may have occurred at the beginning and end of the recording and so that providing a stable signal from the recording. 

The subject was supposed to compare each pair of signals respectively and evaluate whether they cause the same or different perception. Before each synthesized signal, the recorded signal is displayed for 10 seconds. Afterwards the artificially produced audio is displayed again for 10 seconds and the participant was asked to evaluate directly. Hereby, all methods were judged regarding to the reference.

Another important aspect of this experiment is to find the optimum number of frequencies to be selected experimentally as mentioned in section \ref{majorfrequency}. To do this, the synthesized data via 2 to 30 selected frequencies for all materials were prepared to display. We always started with the signal created with 30 frequencies and if the user can not detect any difference between the reference and comparison vibration, the 2-frequency signal was displayed. Gradually from top and bottom the audio signals are displayed via the voice coil actuator, we tried to find the minimum number of frequencies that generate perceptually similar signal to the reference. According to the user's feedback, the optimal number of frequency was evaluated and this result was used for the Experiment II. 

This setting was repeated for each 5 materials and an experiment run was terminated after the subject judged the signal pairs to feel the ``same'' or ``different'' for each material and for major frequency method after determining the optimal number of frequencies to be selected.

\subsection{Results}


\subsection{Conclusion}

\chapter{Tactile Signal Speed Dependency Evaluation}


lpc uses the autocorrelation method of autoregressive (AR) modeling to find the filter coefficients.

\section{Experiment II}
The
four virtual surfaces were created by altering the texture
vibrations? responsiveness to user force and speed. The



\subsection{Subjects and Setup}
from dragging a Phantom Omni across a textured surface [9].
They recorded the spectrum for two tangential speeds and in-


\subsection{Procedure}
Subjects were also given time to practice
using the Omni by exploring a simple haptic environment
consisting of a sphere and cube inside of a box.
--Participants answered ?same? or ?different? with two alternative forced choices to compare the standard and comparison vibration in the aspect of tactile sensation on the finger pad. All participants were instructed to say ?different? if they were reasonably sure that the comparison vibration was different from the standard vibration in the aspect of tactile sensation and ?same?  If participants perceived a difference between the standard and the comparison vibration, they are presented with the same stimuli and comparison vibration once more. 

\subsection{Results}

\subsection{Conclusion}
---

---
 Von Quellen:

We explain how to apply the mathematical principles of Linear
Predictive Coding (LPC) to develop a discrete transfer function that represents the acceleration response
under specific probe-surface interaction conditions. We then use this predictive transfer function to generate
unique acceleration signals of arbitrary length. In order to move between transfer functions from different
probe-surface interaction conditions, we develop a method for interpolating the variables involved in the
texture synthesis process. Finally, we compare the results of this process with real recorded acceleration
signals, and we show that the two correlate strongly in the frequency domain.


These vibrations depend on the motions of the tool and respond
to both normal force and tangential speed. This paper explores
various methods of simulating haptic texture interactions by
rendering tool vibrations that are based on recorded data. We
designed and ran a human-subject study (N=15) to analyze the
importance of creating virtual texture vibrations that respond
to user force and speed. Our analysis of data from fifteen
textures showed that removing speed responsiveness did cause
a statistically significant decrease in perceived realism, but
removing force responsiveness did not. This result indicates
that virtual textures aiming to simulate real surfaces should
vary the rendered vibrations with user speed but may not need
to vary them with user force.
that represents the acceleration response under specific probe-
surface interaction conditions. 

statistically significant decrease in realism
this study elucidated the conditions necessary
to create realistic haptic textures.
 a process of synthesizing probe-surface
interactions from data recorded from real interactions.
via automated analysis of real recorded
data.
While haptic feedback is known to increase the immersion into
a virtual environment (VE), most haptic feedback devices lack
the ability to display multidimensional tactile impressions.
To provide a more efficient and robust method of building haptic texture models from tool-surface interaction data...



 

\begin{equation} \label{eq:isim}
mRG = {\beta \cdot \sum_{k=1}^K \sum_{l=1}^L \hat{\textbf{X}}(k,l)}
\end{equation}

Durch die \texttt{\bslash label} kann auf die Bilder mit
\texttt{\bslash ref} verwiesen werden.


\section{Beispiele f�r Referenzen}
Die Literaturhinweise werden im Text z.B.\ folgenderma�en verwendet:\\
``..., wie in gezeigt, ...'' oder ``... es gibt
mehrere Ans�tze \cite{arnaud99,griswold90} ...''

\section{Schrifttypen}
Als Schrifttyp wird Arial oder Roman empfohlen. Bitte beachten, da�
Gr��en und Einheiten eine eigene Schreibweise haben:
\begin{description}
\item[Kursivschrift:] physikalische Gr��en (z.B.~$U$ f�r Spannung),
  Variablen~(z.B.~$x$), sowie Funktions- und Operatorzeichen, deren
  Bedeutung frei gew�hlt werden kann (z.B.~$f(x)$)
\item[Steilschrift:] Einheiten und ihre Vors�tze (z.B.~kg, pF),
  Zahlen, Funktions- und Operatorzeichen mit feststehender Bedeutung
  (z.B.~sin, lg)
\end{description} 

\clearpage

\section{Archivierung}
F�r die Archivierung sind alle Dateien der Arbeit (auch der Vortr�ge)
dem Betreuer zur Verf�gung zu stellen.  Weiterhin soll noch ein
\BibTeX-Eintrag der Arbeit erstellt werden (die Felder in eckigen
Klammern sind dabei auszuf�llen):
\begin{verbatim}
@MastersThesis{<Nachname des Autors><Jahr>,
  type =         {<Art der Arbeit>},
  title =        ,
  school =       {Institute of Communication Networks~(LKN),
                  Munich University of Technology~(TUM)},
  author =       {<Nachname des Autors>, <Vorname des Autors>},
  annote =       {<Nachname des Betreuers>, <Vorname des Betreuers>},
  month =        {<Monat>},
  year =         {<Jahr>},
  key =          {<Mehrere Suchschl�ssel>}
}
\end{verbatim}
